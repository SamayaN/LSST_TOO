
% TEMPLATE FOR SCIENCE WHITE PAPERS FOR LSST TOO FOLLOW-UP OBSERVATIONS
% VERSION 1.1
% 2011 JAN 18

%-------------------------------------------------------------------------------------------

\documentclass [11pt]{article}
\usepackage{color}
%\usepackage{psfig}
\textwidth 6.55truein
\textheight 9.3truein
\pagenumbering{arabic}
\oddsidemargin 0truecm
\topmargin -2.0cm
\normalbaselines

\newcommand{\samaya}[1]{\textcolor{blue}{[#1]}}
\newcommand{\ashish}[1]{\textcolor{red}{[#1]}}
\newcommand{\andy}[1]{\textcolor{green}{[#1]}}

%-------------------------------------------------------------------------------------------

% PUT ANY SPECIAL LATEX DEFINITIONS HERE: 

\def\chandra{{\it Chandra\/}}
\def\hst{{\it {\it HST}\/}}
\def\spitzer{{\it Spitzer\/}}

\def\simgt{\lower 2pt \hbox{$\, \buildrel {\scriptstyle >}\over {\scriptstyle \sim}\,$}}
\def\simlt{\lower 2pt \hbox{$\, \buildrel {\scriptstyle <}\over {\scriptstyle \sim}\,$}}

%-------------------------------------------------------------------------------------------

\begin{document}

%-------------------------------------------------------------------------------------------

\centerline {\Large Science White Paper for LSST TOO}
%\centerline {\Large Gravity Wave and High-Energy Neutrino Observations}

\vspace*{0.2 cm}

%-------------------------------------------------------------------------------------------

% YOUR EDITING BEGINS HERE
%
% NOTE: In the writing below, you should aim to make your ideas understandable 
%       to a general professional astronomer, rather than just to a specialist 
%       in your specific field. 

\centerline {\Large Gravity-Wave and Particle Events: Targets Of Opportunity}
% Put your white paper title here

\vspace*{0.4 cm}

\noindent 
{\bf Authors:} {\bf ALPHABETICAL FOR NOW} A. Becker (U Washington),  J. S. Bloom (UC Berkeley), K. Cook (LLNL), Z. Ivezic (U Washington), Mansi Kasliwal (Carnegie), A. Mahabal (Caltech), Samaya Nissanke (Caltech) 


\vspace*{0.2 cm}

\noindent 
{\bf Contact Information for Lead Author/Authors:} 601 Campbell Hall, Berkeley, CA 94720; jbloom@astro.berkeley.edu; +1-510-643-3839
% Include postal address, email address, phone number

\vspace*{0.2 cm}

%-------------------------------------------------------------------------------------------

\section{Science Goals}

% Nominal length for this section is about 2 pages. 

\subsection{Concise List of Main Science Goals}

%  * Provide a list of the most important science goals/questions that
%    you are aiming to address. Each list item should be concise with 
%    about 1-3 sentences. 

\begin{enumerate}

\item
Discover and characterize the electromagnetic counterparts to gravitational wave and astrophysical neutrino events. This will require {\it breaking prescribed cadences} of LSST at least a few times a year (perhaps up to a few times a month) for several hours over the course of 1--3 days.

\item Given that this Target of Opportunity Triggering \& Cadencing channel should not impact LSST hardware, we think that opening up this channel for LSST should be seriously considered. The total time commitment of LSST to such a strategy should be small (of order 1\%) and should depend on the variety of GW/neutrino triggering facilities in existence at the time of the science commissioning. 

%This is the second main science goal. 

\end{enumerate}

\subsection{Details of Main Science Goals}

% * Detail the main science goals are you aiming to accomplish with your 
%   proposed deep-drilling observations. 
%
% * Why are these goals of importance, both in your specific area and 
%   in a broader scientific context? Why are they likely to remain
%   important throughout the period of LSST operations?  
The EM/gravity wave connection is an active area of theoretical pursuit and it is likely that in the next several years, the expected EM signatures from the panoply of events that gives rise to GW events will become more clear. However, at this stage, we feel it is important for LSST to build in the capability to 1) slew rapidly to a field with little to no prior notice and 2) build in a dynamic cadence capability -- that could last intermittently for days or even weeks --- for these high priority events. The total time commitment of LSST to such a strategy should be small (of order 1 - 3\%) and should depend on the variety of GW/neutrino triggering facilities in existence at the time of the science commissioning. Aside from the scientific rationale for opening up these capabilities we see important programmatic reasons to do so. LSST is essential unique in connecting the GW universe to the EM universe. It could greatly enhance the returns of the NSF-funded LIGO. Given that this Target of Opportunity Triggering \& Cadencing channel should not impact LSST hardware, we think that opening up this channel for LSST should be seriously considered.

For the vast majority of correlated time-domain science goals, either the nominal LSST cadences will be sufficient or, for events requiring dense and deep sampling, large aperture, small fields of-view facilities will be more appropriate as a followup engine. There are some key Decadal impetuses, however, where the LSST combination of {\bf instantaneous field-of-view, aperture, speed, and wavelength coverage will be critical}. In particular, there is essentially unmatched capabilities of LSST in finding and identifying electromagnetic counterparts to gravity wave and neutrino events. Some of the science cases were articulated in LSST-member-driven whitepapers for Astro2010 (see Bloom et al. 2009 and Kulkarni et al. 2009 and references therein). We summarize here the two main domains of current interest:

\medskip

\noindent {\bf Events from Merging Stellar Mass Remnants:} GW events from neutron star (NS)-NS or NS-black hole (BH) mergers are expected to be readily detectable by Advanced LIGO (and upgraded VIRGO, the Japanese KAGRA etc.). Advanced LIGO, funded by the NSF, is expected come on line a few years before LSST. At advanced sensitivities for a three detector LIGO-Virgo network, predicted event rates (Abadie et al. 2010) 
for: i). NS-NS binaries range from $0.4$ to $400$ per year (with 40 being the ``realistic'' number) detectable up to distances of
$\sim$several hundred Mpc, ii). NS-10 M$_{\odot}$ binaries range from $0.2$ to $300$ per year (with 10 being the ``realistic'' number) out to distances of $\sim 1-2$ Gpcs, and iii). stellar mass binary black holes (BBH) with total masses $<100 M_{\odot}$ range from $0.4$ to $1000$ per year (with 20 being the ``realistic'' number) detectable up to distances of $\sim$several Gpc. In addition to the GW event masses, the detectable range also depends on the number of detectors, and whether their data streams are analysed coherently. However, the localizations of such events will be poor, where the majority of sources will be detected at threshold. Current estimates are that localizations of 20 - 100 sq. deg (90\% confidence) are achievable (Fairhurst 2010) using a LIGO-Virgo network, and the error regions may be disjoint on the sky (Nissanke et al. 2011). With the addition a fourth or fifth detector in Japan or India to the worldwide network, the localizations decrease in size to 5-10 sq. deg (Schutz 2011, Nissanke et al. 2011, Fairhurst 2012). Finding a precise localization via an EM counterpart is critical for breaking degeneracies inherent in the chirp signal (Nissanke et al. 2009) and understanding the nature of the progenitors/surroundings of the event. Importantly, since GW signatures encode luminosity distance but not redshift, an EM counterpart enables the use of such events as precision standard sirens (Schutz 1986, Holz et al. 2005, Dalal et al. 2006). 

%Independent of the cosmological distance ladder and other systematics that plague traditional standard candles, the possibility of measuring $H_0$ to a few percent accuracy, with only an appeal that General Relativity correctly describes the GW emission and evolution, is a compelling impetus.

Precisely {\it what} will be seen in the EM sector is not known for sure. GW event rates are informed by a potential connection to short-hard gamma-ray bursts (SHBs): it is believed that SHBs are the result of NS-NS mergers or NS-BH mergers (see Bloom et al. 2006; Kochanek and Piran 1986). If these are indeed the progenitors of SHBs, then we might expect a coincident afterglow event with a GW event (whether that event is triggered with a gamma-ray burst observation or not). Such afterglows within the LIGO volume would be bright ($R\approx 18$ for the first 10 minutes) and fade rapidly as $t^{-1}$ to $t^{-2}$.

Regardless of whether a NS-NS merger produces a SHB and irrespective of the binary's orientation, there are strong theoretical motivations to believe that a short lived (peak $\approx -$14 mag) event is inevitable. This should look like a mini-supernova at broadbrush but different in the timescales and color evolution (see Metzger et al. 2010 for a recent treatment). At a typical Advanced LIGO distance (250 Mpc), this means we expect an $R=23$ mag event at peak but it could easily be $R=25$ at peak within the theoretical uncertainties. There could also be a prompt signature (not associated with relativistic outflow producing a GRB) but the details of these signatures have not been explored.
 
\medskip

{\bf Neutrino Events:} Neutrino events localized with IceCube (e.g. http://icecube.wisc.edu) and ANTARES  (e.g.  http://antares.in2p3.fr/) will only be localized to at best a few sq. degrees to within a 1Mpc. It could be of interest for LSST to respond rapidly ($<$ hours) to these events.  {\bf MORE TBD}

\subsection{Supplementary Science}

% * In addition to your main science goals, what other science goals
%   will your proposed observations naturally and effectively accomplish? 

LSST will be asked to observe in places on the sky that are not in the regular schedule for a night. We will be asking LSST to spend more time on those places on the sky (perhaps a total of few hours in 50 sq.~degrees) over a few days, returning every 30 minutes to 60 minutes. While these may be non-optimal locations on the sky (viz moon, airmass), breaking away from the prescribed cadences is better than a zero sum game: that is, the loss of the depth in a certain field is an improvement in other field. 

% ----------

% EXAMPLE OF INCLUDING A TWO-PANEL FIGURE
%
% \begin{figure}[t!]
% \hbox{
% \psfig{figure=dr7-z-histo-pretty1.ps,height=2.4truein,width=3.2truein,angle=-90}
% \hskip 0.1 in 
% \psfig{figuTre=dr7-z-histo-pretty1.ps,height=2.4truein,width=3.2truein,angle=-90}}
% \caption[Example two-panel figure]{\protect\small 
% {\bf (a)} Caption for panel a. 
% {\bf (b)} Caption for panel b.}
% \end{figure}

% ----------

%-------------------------------------------------------------------------------------------

\section{Description of Proposed LSST Observations}
% Nominal length for this section is about 2 pages. 

\subsection{List of Proposed Fields}

% * What specific fields are you proposing to deep drill with LSST? For
%   each, please provide a field name, central RA 2000, central DEC 2000.  
%   If your fields are moving over time (e.g., for some Solar System fields), 
%   so that RA 2000 and DEC 2000 are not applicable, then please describe 
%   the needed pointings as specifically as concisely possible. 

{\it These are not known fields at this time, until a trigger is received.}

\subsection{Motivation for Proposed Fields}

% * Explain why these are the best fields for accomplishing your science 
%   goals. Detail any general criteria that you have used for 
%   field selection. 
%
% * Why do you need this number of fields, and not more or fewer? 

N/A

\subsection{Observing Plan, Cadence, Filters, and Expected Depth}

We can expect Advanced LIGO and IceCube triggers to be supplied to followup facilities within a few minutes (right now, the turn around time is 25 minutes and is human-limited). LSST's ability to follow-up a SMBH merger directly results in latency requirements in LISA data transmission (from all three satellites) of $< 6$ hours a week or so prior to the supermassive BH merger.  

To find an EM counterpart we need to repeatedly observe the error region on the sky to find a {\it newly varying/transient source}.  Previous characterization of the variability of known sources, such as variable stars, novae, and older supernovae should improve the chances of identifying the counterpart of interest. However, over 50 sq. deg. there are likely to be XXXX new sources of variability ({\bf MANSI}), such as flaring M-dwarfs without quiescent counterparts. It may be many days and several nights of spectroscopic followup of candidates before the ``correct'' candidate is identified. By that time the source would have faded below detection threshold for even the most large aperture telescopes. And so, in this manner, we view {\bf LSST as both as discovery engine and the main follow-up facility of such events}: we simply will not know which light curve to focus on until it is too late. This is obviously a different view of LSST (largely as a discovery engine) than  for many transient events.

Afterglows of short bursts are intrinsically faint (about 10--100 times fainter than long-duration soft-spectrum GRBs). However afterglows within the LIGO volume would be bright ($R\approx 18$ for the first 10 minutes) and fade rapidly as $t^{-1}$ to $t^{-2}$. Within 2 hours, the afterglow should have faded to $\sim22-23$mag, too faint for any other wide-field small aperture telescope to discover. Two or three bands of observations ($g$,$r$,$i$, nominally) would be required to confirm the expected color (powerlaw with $f_\nu \propto \nu^{-0.5}$) of the event.

From the neutron-rich outflow we would expect an EM event to peak on a 1 day timescale at around $23-25$ mag. A nominal followup routine by LSST would be: 1) receive a trigger from the GW network with localization. 2) on a timescale of minutes to hours begin a dedicated routine to observe the 90\% confidence region of the event 3) repeat visits on this region every 2--4 hours for the next 3 nights. If a prompt signature other than a GRB afterglow is well motivated theoretically, a rapid slew (< minutes) to the field would be required.  All LSST filters would be of use in constraining the SED of the event.

Measuring and identifying the pre-merger EM signature will require repeat visits on 6--12 hr timescales for ~2 weeks, with the cadences increasing in the few days leading up to the event.
% * Summarize your overall observing plan. For example, over what 
%   timespan/timespans do you plan to gather deep-drilling observations? 
%
% * What observation cadence/cadences do you require? 
%  
% * What LSST filters are you proposing to utilize, and why are these
%   the best filter choices?  
%
% * Do you want to change the balance of exposure time, in each LSST 
%   filter, relative to the plan for the main LSST survey?
%
% * How deep will your stacked observations be in the six LSST filters?         
%   Here, and also for some of the points below, the LSST Exposure Time
%   Calculator at 
%   http://dls.physics.ucdavis.edu:8080/etc4_3work/servlets/LsstEtc.html
%   should be useful. 
%
% * Note that, in the above, you should assume availability of the standard
%   LSST filter set as detailed in the LSST Science Book. Other filters will
%   not be available without major revisions to the way LSST works (for
%   many reasons). Similarly, when considering cadence possibilities, please 
%   note the operational constraints of LSST as detailed in the LSST Science 
%   Book. 

\subsection{Observation-Time Cost}

% * Give your accounting of the total LSST observation-time cost for your 
%   proposed observations. This should show your accounting work, so that
%   an external reviewer can understand the accounting being done. LSST
%   overheads should be included. 

%-------------------------------------------------------------------------------------------
\begin{tabular}{lccccccc}
\hline
Scenario & Event/Obs  & Sky & Num.\ of & Cadence & num  & sequence & req.\\
         & Rate  & Coverage & LSST & & epochs & & time\\
         & [$yr^{-1}$]   &  [deg$^2$]  &   pointings & & & & [hr $yr^{-1}$] \\
\hline\hline
kilonovae & 40/10 & 10--100 & 1--10 & 1 epoch/hr & 24 epochs & ugrizy & 5--50 \\
search \\
\hline
prompt    &    $''$     &   $''$       &   $''$     & 1 hr  &  $\sim$50 & gri  & 10--20 \\
counterpart &        &         &      & + 5--10 epochs  \\
            &        &         &      &  over night     \\
\hline
\end{tabular}

\section{Other Required or Relevant Observations}
%Ashish:

\textcolor{red}{Ashish: Possible information that can go here includes:
networking of existing resources to collect and use relevant information; 
existing follow-up networks like CRTS, Gaia, Pan-STARRS, PESSTO, PTF; 
repositories like Skyalert and VAO's Data Discovery Portal; 
characterization and classification services.
Also a small paragraph on existing ICT undertakings (e.g. efforts outside LSST of EM follow-up of GW events using LIGO-IndIGO data).}

% Nominal length for this section is about 0.5 pages. 

\subsection{Other Required Observations}

% * What other multiwavelength observations are required for achieving 
%   your science goals? That is, you would not be able to achieve your
%   science goals effectively without these data.    
%
% * For the required multiwavelength observations, please provide
%   details on how these will be acquired (if not already available). 
Once a set of candidates are identified, we expect that the much of the available resources (especially on large apertures), likely at optical and infrared wavebands will be used to vet the candidates.


\subsection{Other Relevant Observations}

% * What other multiwavelength observations are helpful for achieving 
%   your science goals, though not strictly required? 
If a credible EM counterpart is found from among the candidates, observations from radio to $\gamma$-ray wavebands will be used to study the EM counterpart and constrain the detail physics of the emission processes.

%-------------------------------------------------------------------------------------------

\section{Specific Needs for LSST}

% Nominal length for this section is about 0.5 pages. 

\subsection{Need for LSST}

% * Motivate why LSST is the best telescope to achieve your proposed
%   science goals. 
%
% * Explain why, by the time of expected LSST operations, some other
%   telescope (e.g., one with a smaller field of view) will not be able 
%   to accomplish the main science goals more economically. 

We believe that by 2019, LSST may be the only suitable facility for finding new and rapidly changing transient events on 10--100 sq degree to 24--25 magnitude. Since the expectation of the peak of a kilonova at 200--400 Mpc is at this faint level, a ToO channel with LSST is indeed feasible.

\subsection{Impact of Deep Drilling}

% * Explain why the LSST main-survey observations cannot accomplish your
%   science goals, and thus how your deep-drilling observations advance 
%   science over what can be done with the main-survey data. 

N/A
%-------------------------------------------------------------------------------------------

\section{Feasibility}

% Nominal length for this section is about 1-2 pages. 

\subsection{General Feasibility}

% * Why are your proposed deep-drilling observations feasible for
%   accomplishing your science goals? 
%
% * What are the expected airmass, sky brightness, and seeing for each 
%   of your proposed fields? 


\subsection{Bright Objects and Extinction}
 
% * Please list the brightest 5-10 objects in each of your fields in the
%   LSST r-band. Are there expected troubles from bleed trails, scattered
%   light, etc. due to these objects? 
%
% * Are there any objects in your fields having usual spectra that might 
%   not cause troubles in the r-band but that will likely cause troubles
%   in other LSST bands? 
%
% * Describe the range of foreground dust extinction across your fields, 
%   and how this was evaluated. Does the extinction vary significantly
%   across your fields? 

N/A

\subsection{Unresolved Feasibility Issues}

% * Are there any remaining feasibility/technical issues needing further 
%   work or further observations? If so, please provide details on plans
%   to resolve these issues. 

%-------------------------------------------------------------------------------------------

\section{Other Issues}

% Nominal length for this section is about 0.5 pages. 

\subsection{Relevance to LSST Commissioning}

%  * How will your proposed deep-drilling observations help the success
%    of the main LSST survey and initial LSST commissioning?
We believe it is critical to maintain the core capability of initiating, without prior warning, a series of observations that break from the prescribe observations planned for that night. This Target of Opportunity (or Director's Discretionary Time) may not, in the end, be used to followup many GW or neutrino triggers. But some unforeseen and poorly localized event will inevitably be found and the collaboration should have the logistical and technical capability to respond as demanded by the science.

\subsection{Other Relevant Information}

%  * Provide any other information you consider relevant
The mechanism (or mechanisms) for injection of ``triggers'' for a predefined program of search and discovery resides ultimately with the technical staff of LSST. Simple emails with multiple humans in the real-time loop will not suffice however. We suspect that standard messaging schemes, such as VOEvent, will be mature enough that the GW and LSST communities can set up appropriate conduits for triggering.

%-------------------------------------------------------------------------------------------

\section{References Cited}

% EXAMPLE FORMAT FOR REFERENCES
%
% \vskip 0.05 in 
% \par\noindent\hangindent=10pt\hangafter=1
% Barlow T.A., Junkkarinen V.T., Burbidge E.M., Weymann R.J., 
% Morris S.L., Korista K.T., 
% 1992, ApJ, 397, 81
%
% \vskip 0.05 in 
% \par\noindent\hangindent=10pt\hangafter=1
% Brandt W.N., Chartas G., Gallagher S.C., Gibson R.R., Miller B.P., 
% 2009, 
% in The Monster's Fiery Breath: Feedback in Galaxies, Groups, and Clusters, 
% American Institute of Physics Press,
% p.~49 (arXiv:0909.0958)

%-------------------------------------------------------------------------------------------

\end{document}

